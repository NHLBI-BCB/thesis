\chapter*{Summary}

The genetic code describes how a sequence of codons on an \abbr{mrna} is
translated into a sequence of amino acids, forming a protein. The genetic code
is physically manifested in the cell as \abbr{trna} molecules, which fall into
several classes of anticodon isoacceptors, each decoding a single codon into its
corresponding amino acid. In this thesis I discuss the central importance of the
codon--anticodon interface to \abbr{mrna}-to-protein translation, and how its
stability is maintained during the life of the cell.

This thesis summarises our research into the control of the abundance of
\abbr{trna}s by \abbr{trna} gene expression in mammalian organisms. I hope to
show that \abbr{trna} gene expression is subject to tight regulation, and that
the abundance of \abbr{trna} molecules is thus kept highly stable even across
vastly different cellular conditions, in marked contrast with the abundance of
protein-coding genes, which is changing dynamically to drive cell function.

The abundance of \abbr{trna} genes defines, to a large degree, the efficiency
with which \abbr{mrna} can be translated into proteins. On the one hand, this
serves to explain the need for the observed, stable \abbr{trna} abundance. On
the other hand, this also raises questions: the change of expression of
protein-coding genes means that different, specifically highly expressed
protein-coding genes in different cell types will lead to a different codon
demand. It could thus be beneficial for the cell to express different sets of
\abbr{trna}s, trading lower overall efficiency for high efficiency in
translating the most important subset of genes. I will investigate this
possibility by looking for subtle patterns in gene expression data from
different biological conditions in several mammalian species.
