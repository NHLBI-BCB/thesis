\chapter*{Summary}

The genetic code describes how a sequence of codons on an \abbr{mrna} is
translated into a sequence of amino acids, forming a protein. The genetic code
manifests itself physically in the cell in the form of \abbr{trna} molecules,
which fall into several classes of anticodons, each decoding a single codon into
its corresponding amino acid. In this thesis I discuss the central importance of
the codon--anticodon interface to translation, and how its stability is
maintained during the life of the cell.

The thesis summarises research into the control of the abundance of
\abbr{trna}s from \abbr{trna} gene expression in mammalian organisms. I show
that \abbr{trna} gene expression is subject to tight regulation, and that the
abundance of \abbr{trna} molecules is highly stable even across vastly different
cellular conditions, in marked contrast with the abundance of protein-coding
genes, which are dynamically changing to drive cell function.
