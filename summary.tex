\chapter*{Summary}

The genetic code describes how a sequence of codons on an \abbr{mrna} is
translated into a sequence of amino acids, forming a protein. The genetic code
manifests itself in the cell as \abbr{trna} molecules, which fall into several
classes of anticodon isoacceptors, each decoding a single codon into its
corresponding amino acid. In this thesis I discuss the central importance of the
codon--anticodon interface to \abbr{mrna}-to-protein translation, and how its
stability is maintained during the life of the cell.

This thesis summarises my research into the control of the abundance of
\abbr{trna}s by individual \abbr{trna} gene expression changes in mammalian
organisms. I will show that \abbr{trna} gene expression is subject to tight
regulation, and that the abundance of \abbr{trna} molecules is thus kept highly
stable even across vastly different cellular conditions, in marked contrast with
the abundance of protein-coding genes, which changes dynamically to drive cell
function.

The abundance of \abbr{trna} genes defines, to a large extent, the efficiency
with which \abbr{mrna} can be translated into proteins. On the one hand, this
serves to explain the need for the observed, stable \abbr{trna} abundance. On
the other hand, this also raises questions: the change of expression of
protein-coding genes means that different, specifically highly expressed
protein-coding genes in different cell types will lead to a different codon
demand. It could thus be beneficial for the cell to express different sets of
\abbr{trna}s, trading lower overall efficiency for high efficiency in
translating the most important subset of genes. To investigate this, I examine
the link between \abbr{mrna} expression and \abbr{trna} abundance in a variety
of biological conditions across several mammalian species, establishing that
changes in the pool of \abbr{trna}s are not correlated with changes in
\abbr{mrna} expression.

Overall, my thesis provides important insight into the interface between
transcription and translation, suggesting strongly that the regulation of
translation is weaker than that of transcription in mammals.
