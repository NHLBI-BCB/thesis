\chapter{\abbr{rna} sequencing quantifies protein-coding gene expression}

\section{The transcriptome reflects the state of the cell}

Cells are complex, self-contained units of life, with a cell membrane made up of
a lipid bilayer, which defines a boundary between the outside world and
everything that belongs to the cell. Inside the cell, a highly complex machinery
made up of many parts keeps the cell alive.

As we have seen in the introduction, the central part of this machinery is
abstracted by the Central Dogma of molecular biology, with the \dna at one end
encoding the hereditary identity of the cell, and the proteins at the other end
as the effectors of this information.

It is therefore proteins which determine how a cell behaves, and the change of
proteins ultimately determines changes in cellular function. The entirety of
protein abundance in the cell at a given instance is called the
\define{proteome}. Unfortunately, quantifying the proteome in an unbiased
fashion is hard and expensive\todo{reference}. Instead, modern biology uses the
abundance of \mrna molecules, which code for individual proteins (the
\define{transcriptome}), as an accurate proxy of the proteome. The
appropriateness of this approach has been verified in numerous
studies\todo{citations}.

%\section{Regulation of protein-coding genes}

\section{Microarrays}

Historically, the abundance of \mrna has been determined using specific probes,
which hybridise to complementary target \mrna sequences. In order to determine
the abundance of many different \mrna[s] simultaneously, large arrays of
different probes can be generated and queried in parallel. Due to their
miniaturisation, these arrays are known as \define{expression microarrays}.

While microarrays are still playing an important role in transcriptome analysis,
they have recently been superseded by another technology due, mainly, to two
disadvantages \citep{Casneuf:2007,Marioni:2008}:
\begin{enumerate*}
    \item Each probe is sequence specific, and only recognises its target \mrna.
        As a consequence, quantification is inherently biased and requires that
        each target is known beforehand. Microarrays thus cannot discover new
        target transcripts, and the design of a new microarray design is
        technically challenging and expensive.
    \item Cross-hybridisation causes low-level, non-specific binding of
        transcripts to non-targeted probes, skewing their reported expression
        strength. Since this effect is probe dependent, this means that
        while identical transcripts’ relative abundances can be compared across
        arrays, abundance of different transcripts \emph{on the same array}
        cannot be compared.
\end{enumerate*}

\section{\abbr{rnaseq}}

In 2008, focus shifted from microarrays to whole-transcriptome shotgun
sequencing for \rna quantification \citep{Mortazavi:2008,Marioni:2008}. In
contrast to other \rna quantification approaches, whole-transcriptome shotgun
sequencing (now typically known as \rnaseq) is entirely unbiased in that it does
not rely on a pre-selected set of transcripts to assay. The approach has been
shown to yield high-quality results, is highly replicable, has very low noise,
and is sensitive to transcripts present at low concentration (with a typical
protocol, even single transcripts are often picked up).

\subsection{Biological background of \abbr{rnaseq}}

\rnaseq analysis starts with the enrichment of relevant transcripts from a
sample’s total \rna pool. This is important since the \rna fraction that we are
usually interested in is dwarfed in abundance by the fraction of \rrna (more
than \num{80} per cent of the cell’s \rna is \rrna), and would thus dilute the
signal. This enrichment can be done in either of two ways:
\begin{enumerate*}
    \item Polyadenylated \rna can be targeted due to their affinity to oligo(dT)
        primers, short strings of thymine; this will efficiently capture \mrna,
        which is post-transcriptionally \threep tagged with poly(A) tails.
    \item The \rna can be \rrna depleted using a RiboMinus protocol, which
        specifically targets \rrna molecules for removal.
\end{enumerate*}
Thus, if one wants to profile non-\mrna molecules, \rrna depletion rather than
poly(A) selection is the method of choice.

The enriched \rna is subsequently fragmented to a uniform length of about
\SI{200}{nt}, and reverse transcribed into \cdna. The order of these two steps
may vary, as both variants have advantages and disadvantages. Ultimately,
though, the result is a \cdna library of fragments which are then sequenced on a
high-throughput sequencing machine (\cref{fig:rna-seq-procedure}).

\textfig{rna-seq-procedure}{spill}{\textwidth}
    {\rnaseq procedure.}
    {Adapted from \citet{Mortazavi:2008}.}

\subsection{Computational methods to analyse \abbr{rnaseq} data}

Sequencing the \rnaseq samples yields short-read libraries several tens of
millions of reads in size, with reads of uniform length from \SI{25}{bp} up to
(currently) about \SI{125}{bp}. These are then mapped to a reference --- either
the whole genome or the transcriptome --- or assembled \emph{de novo} (usually
in absence of a suitable reference). This assigns reads to genomic locations,
which can be queried and matched to features (usually genes or exons).

One important distinction depends on whether the \cdna fragments were sequenced
from both ends or from one end only. In the first case, instead of a single read
per fragment we end up with paired-end reads, which are separated by a known
distance. Mapping and assembling paired-end reads creates further algorithmic
challenges but increases the amount of information contained in a read (pair),
which increases the amount of unambiguously mappable data.

\subsection{Expression normalisation}

Since the end goal of \rnaseq is to quantify transcript abundance, the next step
is to quantify the number of reads mapping to genomic features. In the simplest
case, one can count the number of reads overlapping with a feature’s genomic
range. This is what e.g.\ \name{HTSeq} \citep{Anders:2014} does. However, read
counts obtained in this manner vary with the length of the mappable region ---
longer features originate more sequenced fragments, and hence more reads, with
equal coverage, compared to shorter features --- and with the total size of the
sequenced library.

\cite{Mortazavi:2008} therefore introduced a relative measure of transcript
abundance, the \rpkm, defined as

\begin{equation}
    x^*_i = \frac{x_i}{\tilde l_i \cdot 10^{-3} \cdot n \cdot 10^{-6}}
        \text{\ ,}
\end{equation}

where \(x^*_i\) is the \rpkm of transcript \(i\), \(x_i\) is the read count on
the transcript, \(\tilde l_i\) is the effective length of the transcript (i.e.\ 
its length minus the fragment length plus \num{1}) and \(n\) is the library size
(in number of reads). With the advent of paired-end sequencing this measure has
been supplanted by the \fpkm, simply replacing the number of reads in the
equation with the number of fragments (i.e.\ the read pairs rather than single
reads in the case of paired-end data).

A related measure is the \tpm, which additionally normalise by the total
transcript abundance \citep{Li:2010}. In other words,

\begin{equation}
    x^*_i = \frac{x_i}{\tilde l_i} \cdot \left(\sum_j{\frac{x_j}{\tilde
        l_j}}\right)^{-1} \cdot 10^6 \text{\ .}
\end{equation}

\tpm succinctly answers the question, “given one million transcript in my
sample, how often will I see transcript \(i\)?” It is important to note that
both approaches give a sample dependent abundance, and thus neither of
these units makes measured transcript abundance comparable across different
experiments. To do this, a different approach has to be taken.

\subsection{Differential expression}

\section{Using \abbr{rnaseq} to assay gene expression during mouse development}

\todo{This describes the specific method used in the \abbr{trna} paper}
In order to investigate protein-coding gene expression, we quantified the \mrna
abundance from \rrna-depleted \rnaseq data (strand-specific \SI{75}{bp}
paired-end reads from \name{Illumina} \name{HiSeq~2000}). Reads were mapped to
the \mmu reference genome (\abbr{ncbim37}) using \name{iRAP}
\citep{Fonseca:2014} and \name{TopHat2} \citep{Kim:2013}. Read counts were
quantified using \name{HTSeq} \citep{Anders:2014}, and assigned to
protein-coding genes from the \name{Ensembl} release \num{67}
\citep{Flicek:2014}.\todo{add \chipseq and \rnaseq pipeline flowcharts}

We excluded mitochondrial chromosomes from the analysis, because mitochondrial
genes use a slightly distinct genetic code\todo{ref}. Furthermore, we excluded
sex chromosomes.
