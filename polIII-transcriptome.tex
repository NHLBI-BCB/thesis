\chapter{The \abbr{pol3} transcriptome consists of more than just \abbr{trna}}
\label{sec:pol3}

\section{A profile of \abbr{pol3} binding across different features}

Less than \num{5} per cent of the human genome is protein-coding. Much less than
\num{1} per cent of the genome codes for \trna[s] \citep{Lander:2001}.
Nevertheless, \pol3 can transcribe a variety of \rna species, with \trna genes
forming only a part of the overall \pol3 transcriptome. Notable other targets of
\pol3 transcription are the 5S \rrna (a part of the ribosome) as well as the U6
\lowsc{sn}\rna which forms part of the spliceosome, a large riboprotein that is
implicated in the post-transcriptional processing of \rna[s] \citep{White:1998}.
In this chapter I briefly interrogate other members of the \pol3 transcriptome.

For this analysis, I re-examined the \chipseq data in the developing liver of
\mmu from \cref{sec:trna}. I was interested in generating a profile of how much
binding of \pol3 occurs at different functionally annotated regions in the
genome. \Cref{sec:trna} examined just one such region, the \trna genes. To
compare this with the amount of binding in other genomic features, I quantified
the \chipseq reads that mapped to annotated features from the \abbr{grcm38}
mouse genome annotation curated by \name{Ensembl} (release \num{75})
\citep{Flicek:2014}. Furthermore, I used data from annotated repeats because it
is known that \pol3 binds to, and potentially drives transcription of, several
types of retrotransposons \citep{Carriere:2012}, which are screened and
annotated by \name{RepeatMasker} \citep{Smit:2014}. \abbr{Pol3} \chipseq reads
were mapped to the \mmu reference genome \abbr{grcm38} using \name{Bowtie}
\citep{Langmead:2009}.

As explained in \cref{sec:chip}, many \trna genes arise through gene duplication
events and we thus expect many reads to map to multiple locations. This problem
also exists for the other annotation types we are interested in. However, the
strategy also explained in \cref{sec:chip} cannot be applied to non-\trna
annotations since we cannot make the same assumptions about the binding profiles
in the flanking regions of the genomic features. In particular, while \trna
transcription uses a type \abbrsc{II} transcription initiation, other \pol3
targets use different types of initiation, due to their different promoter and
enhancer structure. These differences could have strong effects on the binding
profile of active \pol3 on the target loci.

I overcame this challenge by only reporting a single match per read, even if
multiple matches were possible. This assigns the read to an arbitrary locus
amongst its possible matches. As long as all potential match positions for a
read fall into the same type of annotation, this should not pose a problem for
the analysis: all we are interested in is which annotation type a read falls
into, not where on the genome.

The annotations used in this analysis required some manual curation. Excluded
from the subsequent analysis are repeat types “simple repeats”, which stands for
small repeats of a few bases, and “tandem repeats”, where the variability of the
copy number between the reference and the sample makes it impossible to
quantify the \chip signal. Furthermore, I merged other types of repeats, except
those corresponding to retrotransposons and those corresponding to other gene
copies (such as \trna[s]), which I counted separately.

I then counted reads overlapping with each of the annotations mentioned above
and calculated \tpm[s]. \Cref{fig:pol3-coverage} compares the abundance of \pol3
binding on different features. As expected, \trna accounts for a majority of the
total binding. Unfortunately, this makes it hard to assess the remaining
variability visually. The remainder of the data is therefore summarised again in
tabular format, for just one stage (\cref{tab:pol3-coverage}).

\textfig{pol3-coverage}{spill}{\textwidth}
    {Polymerase \abbrsc{III} coverage,}
    {compared across different feature types in six stages of development in
    liver. Strikingly, the E18.5 stage shows strongly reduced overall \trna
    activity. This is due to a single, divergent replicate, which pulls down the
    average.}

\begin{table}[!ht]
    \centering
    \begin{tabular}{@{}lr@{}}
        \toprule
        Feature & {Prop (\%)} \\
        \midrule
        \abbr{rrna} & 31.1 \\
        \abbr{transsine} & 11.7 \\
        \abbr{ncrna} & 10.6 \\
        repeat & 10.3 \\
        \abbrsc{LTR} & 10.2 \\
        pseudogene & 9.7 \\
        protein-coding & 9.3 \\
        \abbr{transline} & 7.2 \\
        \bottomrule
    \end{tabular}
    \tabcap{pol3-coverage}{Polymerase \abbrsc{III} coverage}{excluding
    \abbr{trna} genes for liver E15.5.}
\end{table}

Interestingly, the proportions with the extreme skew towards \trna genes shown
in \cref{fig:pol3-coverage} are similar to those reported in
\citet{Raha:2010,Canella:2012} for \pol2 association with \pol3 genes and
repeats. As can be seen in \cref{tab:pol3-coverage}, with the exception of \rrna
and \transline[s], all features have very similar coverage proportions. These
numbers are probably skewed by the \pol3 \chipseq background signal, which has
not been removed from the data, on the assumption that, as long as all features
are susceptible to the same amount of spurious signal, the influence on the
proportions would be negligible. Comparing the signal strength from the input
libraries and the \pol3 \chip reveals that this may not be the case (see
\cref{fig:pol3-inputs}). Consequently, in the future it will be important to
verify that the results described below are robust after properly modelling
background noise levels.

\subsection{\abbr{transsine}s are transcribed by \abbr{pol3}}

Taken together with the observation reported by \citet{Carriere:2012}, the
results compelled me to take a closer look at the binding of \transsine[s] by
\pol3. \transsine[s] are a type of retrotransposon that are highly abundant in
the genome: as much as \num{13} per cent of the genome is a \transsine in
mammals, and there are of the order of \num{1500000} gene copies
\citep{Lander:2001}. \transsine[s] are are anywhere from \SIrange{100}{700}{bp}
in length and are typically composed of three elements: the \define{head}, the
\define{body} and the \define{tail}. The head of \transsine[s] is derived from
\pol3-transcribed \rna[s]. The body is an unrelated piece of sequence often
containing fragments from \transline[s]. The tail, finally, usually consists of
a simple repeat.

In mammals, most \transsine[s], which form the class of \define{Alu} elements,
are derived from the (\pol3-transcribed) 7SL \ncrna. Many others, forming
different classes of \transsine[s], are derived from different \trna genes
\citep{Vassetzky:2013}.

The fact that they possess a \pol3 promoter means that they can, in some cases,
be transcribed in vitro \citep{White:1998}. It is generally assumed that they do
not perform a function in the cell and are usually not actively transcribed.
However, as the results by \citet{Carriere:2012} indicate that there is indeed
limited transcription of \transsine in mammals, we should be able to observe
this in our \pol3 \chip data. It is my hope that in examining \transsine
transcription, we may be able to learn more about the transcription regulation
of \trna genes: the similar promoter structure of \trna-derived \transsine[s]
suggests that the mechanism of gene regulation may be similar here. Furthermore,
since \transsine[s] are not evolutionarily conserved and perform no vital
function in the cell, their active transcription will always happen
incidentally. Comparing the upstream sequence composition of active and inactive
\transsine[s] and of related \trna genes may enable us to pinpoint common
sequence motifs necessary for active transcription.

Testing this is not trivial, since \transsine[s] are repeat elements and we thus
cannot directly map reads to unique locations, similar to the case of \trna
genes. In the following, rather than looking at the flanking sequences to
disambiguate non-uniquely mapping reads as I did for \trna, I pooled classes of
\transsine gene copies into \num{676} classes (downloaded from \name{RepBase}
\citep{Jurka:2005}). I then generated a reference transcriptome from the
consensus sequences of the \transsine classes. Reads from \pol3 \chipseq were
then mapped to this reference using \name{Bowtie}.

Across two tissues (liver and brain) and six stages of development I find that
on average \num{63} per cent of the \transsine classes (\num{431} out of
\num{676}) show nonzero coverage. In these, there is low but significant
enrichment of \pol3 binding compared to the input libraries (\chipseq experiment
performed without antibody to quantify the noise introduced by the experiment).
The amount of \pol3 binding to the various \transsine classes varies across
\num{4} orders of magnitude. An overview over the magnitude of binding is shown
in \cref{fig:sine-summary}.

\textfig{sine-summary}{spill}{\textwidth}
    {\abbr{transsine} binding by \abbr{pol3} across development in liver and
    brain.}
    {Raw counts of \abbr{pol3} binding to different \abbr{transsine} classes.
    Classes with count equal to \num{0} have been filtered out.}

Despite the challenges of working with repeat regions, the quantification of
changes in \transsine binding by \pol3 during mouse development seems thus
feasible. The next problem to solve is the handling of experimental noise, shown
in the input libraries. Conventional \chipseq protocols for the identification
of protein binding sites or histone modifications use these libraries during
peak-calling to quantify and filter out the background signal
\citep{Zhang:2008}. The naive way of handling this, simply subtracting the input
count from the \pol3 count, will yield negative values in many cases where the
signal is low and noise is high (about \num{40} per cent in this dataset).

After accounting for noise, the remaining expressed \transsine classes can be
classified based on the \rna they are derived from. Different classes of
\transsine have a highly variable number of gene copies, from
\numrange{100}{1000000}. When comparing \pol3 binding between these classes,
it is important to account for these differences, since higher gene copies will
have proportionally high\-er binding.

\parrule

In summary, I have shown that \trna genes account for the vast majority of the
overall \pol3 binding. After this, \rrna genes show the most binding, and
following this, \transsine[s]. I have mapped the \pol3 \chip data against
consensus sequences for \num{676} \transsine classes to show that many of these
classes have nonzero \pol3 binding. In the future, I plan to investigate how
\pol3 binding varies between different classes of \transsine[s], in particular
with regards to their origin. Furthermore, I plan to investigate how this
binding changes across development, and whether this variation in binding
correlates between any of the \transsine classes and related \trna genes.
Finally, if such a correlation is found, I will look for common upstream
sequence elements that are absent in other \transsine classes. However, this
last step may require looking at the upstream sequence of individual
\transsine[s] rather than that of \transsine classes unless it turns out that
the upstream sequences of individual \transsine[s] within a class is conserved.

I have focused on \transsine[s] due to their interesting similarity to \trna
genes. A similar analysis as for \transsine[s] can of course also be performed
for other repeat classes that showed similar levels of \pol3 binding in the
previous analysis. In fact, there is potential for the \transsine analysis to
yield a framework to facilitate this kind of sequencing analysis for repeat
features, which are notoriously hard to work with in the context of
high-throughput sequencing due to their inherent non-mappability.
