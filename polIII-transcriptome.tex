\chapter{The \abbr{pol3} transcriptome consists of more than just \abbr{trna}}
\label{sec:pol3}

Less than \num{5} per cent of the mammalian genome, in terms of sequence length,
is protein coding. Much less than \num{1} per cent codes for \trna[s]. Indeed,
most of the genome as we know it is untranscribed\todo[ref]{}. Nevertheless,
\pol3 transcription can occur at a wide range, and \trna genes form only a part
of the overall \pol3 transcriptome. In this chapter we are going to take a look
at other members of the \pol3 transcriptome.

For this analysis, I re-examined the \chipseq data in the developing liver of
\mmu from \cref{sec:trna}. I was interested in generating a profile of how much
binding of \pol3 occurs to different functionally annotated regions of the
genome. \Cref{sec:trna} examined just one such region, the \trna genes. To
compare this with the amount of binding in other genomic features, I quantified
the \chipseq reads that mapped to annotated features from the \abbr{grcm38}
mouse genome annotation curated by \name{Ensembl} (release \num{75})
\citep{Flicek:2014}. Furthermore, I used data from annotated repeats because it
is known that \pol3 binds to, and potentially drives transcription of several
types of retrotransposons \citep{Carriere:2012}, which are screened and
annotated by \name{RepeatMasker} \citep{Smit:2014}.

As explained in \cref{sec:chip}, many \trna[s] are results of gene duplications
and we thus expect many reads to map to multiple locations. This problem also
exists for the other annotation types we are interested in. However, the
strategy also explained in \cref{sec:chip} cannot be applied to non-\trna
annotations since we cannot make the same assumptions about the binding
profiles in the flanking regions of the genomic features. In particular, while
\trna transcription uses a type \abbrsc{II} transcription initiation, other
\pol3 targets use different types of initiation, due to their different promoter
and enhancer structure. These differences could have strong effects on the
binding profile of active \pol3 on the target loci.

I avoided this problem by only reporting a single match per read, even if
multiple matches were possible. This assigns the read to an arbitrary locus
amongst its possible match hits. As long as all potential match positions for a
read fall into the same type of annotation, this does not pose a problem for the
analysis: all we are interested in is to say which annotation type a read falls
into, not where on the genome.

\todo[inline]{Show distribution of features binding pol3}

Disregarding unannotated regions, we can see that \trna gene transcription
accounts for about \num{25} per cent of the total \pol3 binding across mouse
development. The remainder is split up between other types of non-coding
\rna[s].

… with a special focus on one class
of \define{retrotransposons}: \abbr{transsine}s.

\todo[inline]{Zoom in on SINEs}
