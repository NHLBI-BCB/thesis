\chapter{Analysis challenges concerning \abbr{trna} expression data}
\label{sec:trna-analysis}

\section{Quantifying expression of \mrna genes}

In order to investigate protein-coding gene expression, we quantified the \mrna
abundance from \rrna-depleted \rnaseq data (strand-specific \SI{75}{bp}
paired-end reads from \name{Illumina} \name{HiSeq~2000}). Reads were mapped to
the \mmu reference genome (\abbr{ncbim37}) using \name{iRAP}
\citep{Fonseca:2014} and \name{TopHat2} \citep{Kim:2013}. Read counts were
quantified using \name{HTSeq} \citep{Anders:2014}, and assigned to
protein-coding genes from the \name{Ensembl} release \num{67}
\citep{Flicek:2014}.
\todo{add \chipseq and \rnaseq pipeline flowcharts}

We excluded mitochondrial chromosomes from the analysis, because mitochondrial
genes use a slightly distinct genetic code\todo{ref}. Furthermore, we excluded
sex chromosomes.

%\section{Normalisation of \trna gene expression}
%
%Raw \trna gene expression counts show strong differences between sequencing
%libraries. For \mrna count data, it is customary to use \name{DESeq}’s
%\dfn{library size normalisation} \citep{Anders:2010}. However, as shown in
%\cref{fig:trna-libraries-example}, library size normalised data still follows
%distinct distributions, which we deem biologically implausible, and probably due
%to technical bias.
%
%As a consequence, we opted to \dfn{quantile normalise} the count data\todo{find
%reference for quantile normalisation}.
%
%\textfloat{trna-libraries-example}{body}
%    {\centering\Huge Placeholder}{Placeholder}{Placeholder}

\section{Compensation}

For each isoacceptor that is encoded by more than two \trna genes, we calculated
Spearman’s rank correlation (across developmental stages) between the expression
values of each pair of its corresponding \trna genes, i.e.\ we calculate

\begin{equation}
    c_{ij} = \operatorname{cor}(x_i, x_j) \text{ for \(i, j \in T, i < j\)},
\end{equation}

where \(T\) is the set of \trna genes in the isoacceptor family, and \(x_i\) is
the vector of expression values of the \(i\)th \trna gene across all stages of
development. For the same set of genes, we calculated a null set of correlations
as follows:

\begin{equation}
    b_{ijk} = \operatorname{cor}(\operatorname{perm}_k(x_i), x_j)
        \text{ for \(i, j \in T, i < j; k \in 1\dots\lvert x_i\rvert!\)}.
\end{equation}

Here, \(\operatorname{perm}_k(x_i)\) is the \(k\)th permutation of the vector
\(x_i\).

Next, we used the \(\chi^2\)-test to investigate whether there was a significant
difference between the background \(b\) and the observed correlation
distributions \(c\). We reported the Bonferroni-corrected \(p\)-value for the
\num{27} isoacceptor families with six or more genes, since isoacceptor families
with less than six genes did not enough points for meaningful interpretation.

\section{Genomic clusters}

We defined \num{69} clusters of all genomically annotated \trna genes that lie
within \SI{7.5}{kb} of each other. We counted how many active \trna genes of an
isoacceptor family colocalised in a genomic cluster with \trna genes of the same
isoacceptor family. We calculated the fraction of \trna for each isoacceptor
family belonging to a genomic cluster. In order to test whether genes in
isoacceptor families tend to genomically colocalise more than expected by
chance, we randomly assigned \trna genes to isoacceptor families (preserving the
actual isoacceptor family gene numbers) \num{1000} times. We then tested whether
the mean percentage of clustering \trna genes per isoacceptor family differed
from the mean percentage expected by chance, by using a binomial test. Finally,
we tested whether there wa a difference in these percentages between isoacceptor
families that show evidence for compensation, and isoacceptor families that show
no such evidence by applying a \(\chi^2\)-test.

\section{An overview over failed analysis approaches}
