\chapter{Conclusion}
\label{sec:conclusion}

\dictum{… except in the light of genetics.\par\dictumrule}

\noindent
Evolution by natural selection is the implicit assumption underlying all of
modern biology. \citet{Dobzhansky:1973} famously argues against ill-informed
criticism of evolution, stating

\begin{quote}
    Nothing in biology makes sense except in the light of evolution.
\end{quote}

This statement is as relevant today as it was then --- both as an admonishment
of ignorance of basic biological facts, and as serving as a summary of our
understanding. In fact, modern evolutionary synthesis, which is the prevailing
explanatory model employed today, and which is itself an evolution of the
Darwinian theory of descent with modification by means of natural selection, is
unchallenged in this status, and is corroborated by every new piece of evidence.

In \texttitle{The making of the fittest}, Sean Carroll argues that the best
evidence for evolution we have is the genetic record which we can read in the
\dna of the living species \citep{Carroll:2006}. Carroll was mainly talking
about the similarity between homologous genes in different species. But one of
the most striking examples of strong conservation is the near-universal genetic
code. In billions of years of evolution, the instruction set used to encode the
building blocks of proteins has remained virtually unchanged.

In contrast with the conservation of the genetic code itself, the implementation
of this code shows some degree of variation, most notably in the variation of
the selection of preferential synonymous codons (reviewed in
\citet{Ermolaeva:2001}) on the one hand, and in the divergence of the \trna
genes \citep{Kutter:2011} on the other hand.

In \cref{sec:trna}, I have summarised our research into the variability of the
\trna genes, not across evolution, but across development. Our findings mirror a
common theme: despite pervasive variation of \trna gene expression, the antiodon
isoacceptor \trna abundance remains very stable across different stages of
development, and matches the codon demand of the transcriptome which, itself,
also shows very little variation.

Despite the existence of larger variation of codon usage between subsets of
genes, some of which are cell type specific, we were unable to find evidence for
a regulatory effect of this codon bias on translation rates via higher
adaptation to a cell type specific \trna anticodon isoacceptor pool in mammals.
On the contrary, the variation in the \trna anticodon abundance does not seem to
correlate with cell type specific codon demand. This finding, presented in
\cref{sec:codons} lends support to a view that has recently been challenged,
that translational selection via codon bias, if present at all, plays a
negligible role in mammalian systems. It will be interesting to see how this
controversy will continue.

The \pol3 \chipseq data generated for the projects presented in this thesis
provides a wealth of information beyond just \trna gene activity. \cref{sec:pol3}
takes a brief glimpse at genome-wide \pol3 binding and confirms previous reports
of the association of \pol3 with \transsine loci in vivo. I will investigate
these closer in the future.

\section{Future directions}

\subsection{Regulation of \trna transcription}

While providing unprecedented insight into the controlled variability of \trna gene
transcription, \cref{sec:trna} has failed to establish a mechanism for the
differential regulation of \trna gene transcription. Known features of \pol3
recruitment, such as transcription factor binding and specific histone marks,
could not conclusively shown to cause the differences we observed in the \trna
gene transcription between different stages of development. My analysis
deliberately excluded the internal promoters of \trna genes from the search for
specific motifs since it has previously been reported that variation of the
internal promoter of \trna genes is unrelated to variation in gene expression
\citep{Oler:2010,Canella:2012}.

However, results in \citet{Gingold:2014} indicate that this may have been
premature, as they find significant differences in the B box of \trna genes
which they reported as differentially expressed between different conditions.
This suggests that internal promoter variation may contribute to the observed
variability of the \trna transcriptome after all. I intend to run the methods
they used on our data to test this hypothesis.

\subsection{Codon usage adaptation}

The question of what causes codon usage bias variability across functional
subsets of the transcriptome remains wide open. \gc bias, in particular, is
worth exploring further. On the one hand, I observed a robust correlation
between \gc bias and codon usage, and we know that codon usage can sometimes be
predicted from intergenic \gc bias \citep{Chen:2004}. On the other hand,
\citet{Duret:2002} show that, at least in \species{dmel} and \species{cel}, \gc
bias is uncorrelated with codon usage bias.

To explore this further, two avenues present themselves:

\begin{enumerate}
    \item \citet{Chen:2004} show that intergenic \gc predicts codon usage. Since
        intergenic regions are non-coding, this suggests that differential codon
        usage between genes has, at best, a minor functional relevance. So far,
        I have only looked at \gc bias in coding sequences. To make similar
        conclusions, I will have to instead compare codon usage to the \gc bias
        in the flanking regions of protein-coding genes.
    \item \gc bias can vary between synonymous and non-synonymous codons. If \gc
        bias is indeed causal for codon usage, we would expect that \gc bias
        correlates highly between the first two nucleotide positions of the
        codon and its wobble position. However, if the wobble position’s \gc
        bias is uncorrelated to the \gc bias of the other codon positions in a
        gene set, this would require a different explanation.
\end{enumerate}

Another feature known to constrain codon deployment is the presence of other
sequence features in the coding region of genes. This includes binding sites for
enhancers \citep{Blencowe:2000}.\todo{what to do about it?}

\todo[inline]{Conservation of codon usage and GC bias}

The usage of a simple correlation between matching codons and anticodons,
disregarding wobble base pairing, has proved adequate to demonstrate an overall
high correlation between the codon demand and matching \trna anticodon
isoacceptor pool. However, arguing about the relative adaptiveness of different
gene sets or transcriptomes may make it  necessary to consider wobble base
pairing to model the codon--anticodon interaction more precisely. The \tai
\citep{Dos_Reis:2003} offers a way of quantifying \trna adaptation based on
\trna gene copy numbers for each isoacceptor family. The \tai considers only
simplified wobble rules (\cref{tab:wobble}), rather than the extended rule set
for modified nucleobases (\citep{Murphy:2004}). Future work that is planned for
the analysis presented in \cref{sec:codons} includes adapting the \tai for use
with \trna expression data and using it in preference of a codon--anticodon
correlation coefficient.
