\clearpage \chapter*{Acknowledgements}
\addcontentsline{toc}{chapter}{Acknowledgements}

I have been supported by many people while working on this thesis. I would like
to thank them for their contributions. In particular, thanks go to John Marioni
who made the fatal mistake of accepting to supervise me as his PhD student. He
provided keen insight and scientific support for my research and pushed me to
exceed my own expectations. He also provided extensive and always useful
feedback on the thesis itself. Many thanks to my collaborators of the research
presented here, Bianca Schmitt and Claudia Kutter. Both performed excellent work
in the generation and curation of the data, and only with their expertise was it
possible to develop the research. They are also exceedingly patient for the
quirks of computational researchers. The same goes for their group leader,
Duncan Odom, who expertly complemented John’s tutelage. I would like to extend
very warm thanks to Anja Thormann who, more than anyone else, convinced me to
take a risk, leave my comfort zone and come to study at the University of
Cambridge. She also taught me much about the everyday life in research, and laid
my foundations in statistics (which I sorely neglected during my previous
studies).

A very special thanks goes to my two flat mates, Maria Xenophontos and Nils
Kölling. I believe that I have discussed every single aspect of this thesis
extensively with them, and they have helped me catch many embarrassing mistakes
early. They also taught me the use of the \name{ggplot2} library. Finally, they
also kept kicking me, figuratively, to get to work on my thesis (sometimes more
than strictly necessary, Maria). I would like to thank Myrto Kostadima and Remco
Loos for teaching me to work with \abbr{chipseq} data, and I’d like to thank
Ângela Filimon Gonçalves not only for teaching me much about working with
\abbr{rnaseq} but also for many fruitful discussions which have impacted the
analysis in many positive ways. Of my office mates I would like to thank
Jean-Baptiste Pettit and Nuno Fonseca for extensive discussions about software
tools and productivity. Nuno also helped me circumnavigate many software
problems relating to the \abbr{rnaseq} analysis. Catalina Vallejos provided
excellent support for statistical questions. Antonio Scialdone, Jong Kyoung Kim,
Luis Saraiva, Tim Hu and Nils Eling helped by discussing many aspects of the
analysis. Anestis Touloumis in particular helped me by discussing the design of
the codon compensation analysis.

Michael Schubert and Stijn van Dongen provided fruitful discussions about
software. In particular, Michael’s management of the cluster’s software
installation and Stijn’s many excellent tools saved me a lot of time. Matthew
Davis gave me copious feedback on many different parts of the analysis and in
particular on the chapter on codon--anticodon adaptation. My thesis advisory
committee, consisting of Paul Flicek, Gos Micklem and Detlev Arendt provided
guidance for my research and suggested several new avenues of inquiry.

I am grateful to my parents, Gero and Vera Rudolph, as well as my sister, Sophia
Rudolph, for nurturing my interest in science from early on. They are always
excited about discussing every aspect of my work, and make an effort to really
understand my research, which has led to much useful feedback. For moral support
during the PhD, I’d like to especially thank the \abbrsc{EMBL} predoctoral
community, which consists of many fine individuals and which provides, in equal
measure, scientific feedback, entertainment and psychological counselling.

Finally, I would like to thank Joana Borrego Pinto for many contributions small
and large to my research, in particular help with the design of figures, the
presentation of data and suggestions about the interpretation of the results.
But more than that, for everything else.
