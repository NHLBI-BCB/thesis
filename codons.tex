\chapter{Implications of codon–anticodon interaction on the regulation of translation}

\todo[inline]{Insert section titles}

In the previous chapter I have shown that the codon usage and its interaction
with \trna anticodons remains remarkably stable across the variability of the
transcriptome during mammalian development.

Shortly after the publication of our research on \trna gene regulation during
mouse development, \citet{Gingold:2014} published \texttitle{A Dual Program for
Translation Regulation in Cellular Proliferation and Differentiation}. They
report that different programmes of cellular function preferentially use
different sets of codons, and that the pool of active \trna[s] adapts
dynamically to decode this set of codons with high efficiency.

In order to bring in line these two findings — the stability of the anticodon
pool on the one hand, and the malleability of the anticodon pool to match demand
of highly expressed on the other hand — we turned our attention to differences
between healthy tissue and tumour cell lines in \mmu and \hsa.

\todo[inline]{Summarise \citet{Gingold:2014} here}

\citet{Gingold:2014} investigated the abundance of \trna anticodons and the
codon usage of different groups of protein-coding genes in human-derived cell
lines in different cellular conditions --- N\todo{find number} tumour tissues,
induced differentiation, release from serum starvation and senescence.

Using microarray expression data, they show that there are specific anticodon
isoacceptors whose abundance changes reproducibly in specific tumours
(lymphomas), mirroring results reported previously\todo[ref]{Canella? White?}.
They then looked at genes of \go terms which they associated with healthy, adult
tissue (“pattern specification process”) and tumour tissue (“M phase of mitotic
cell cycle”). They show that the codon usage bias in these two \go terms
(averaged over all containing genes) clearly differs\todo{fig 2A?}.

They expand their analysis by calculating the mean codon usage bias\todo{codon
usage or CUB?} across many
categories of \go terms and performing \pca on them. This reveals that the
largest contributor to the variation stems from the split of the \go terms into
two distinct sets, one encompassing multi-cellular \go terms and the other cell
autonomous \go terms. They argue that these two sets of \go terms correspond to
\go terms functionally responsible for maintaining cell homoeostasis on the one
hand, and rapid cellular division, such as as found in tumours, on the other
hand. In other words, different sets of codons are preferentially used in
rapidly dividing cells than in healthy cells.

\textfig{go-cub-pca}{body}{0.8\textwidth}
    {\pca of mean \go term codon usage bias.}
    {\todo{Add description}. Created using methods of \citet{Gingold:2014}}

\todo[inline]{Include table of two sets of GO terms referred to above}

Finally, \citet{Gingold:2014} use \rnaseq data of the previously mentioned cell
types to argue that gene expression changes do indeed correlate with the first
principal component of the \pca, with tumour samples showing more increased
expression towards the more we go towards the cell autonomous end of the axis,
and more decreased expression towards the multi-cellular end, and that this
trend is reversed for samples taken after induced differentiation using retinoic
acid. A similar analysis is then done for the expected translational efficiency,
by using \trna anticodon abundance to calculate the fold change of the \tai\todo{explain tAI},
matched against the codon usage bias in the different \go terms. However, the
authors neglected to actually calculate correlations between the first principal
component and either the codon usage bias or the \tai fold change. Instead, they
merely visualised the presumed correspondence using an inadequate and
(particularly in this case) misleading colour map \citep{Borland:2007}.
Furthermore, the absolute range of changes are minute (range of \tai fold change
\numrange{0.86}{0.905} in one case), and lack of statistical analysis makes it
impossible to say whether these changes are in fact significant, assuming they
correlate at all\todo{fig 3?}.

Since we were already in possession of relevant \trna data we decided to use our
own data to recapitulate these findings, rather than merely using the data from
\citet{Gingold:2014}. In addition, since we were not convinced by the
statistical analysis presented in their paper, we decided to address this as
well.

A first observation was that while our own research so far had looked at the
whole transcriptome, \citet{Gingold:2014} had looked at specific subsets. While
both approaches are valid in their own right, some caution is necessary when
comparing the results: the smaller the set of genes one looks at, the larger the
effects of random sampling of the genes become. When analysing a particular
feature, such as the codon usage bias, random sampling will thus contribute a
larger part to the variation between two small sets than between two large sets.
If one wants to assert that deviations are nonrandom, one thus has to account
for this effect (\cref{fig:sample-size-dependent-cub}).

\textfig{sample-size-dependent-cub}{spill}{\textwidth}
    {Dependence of codon usage variability on sample size.}
    {Genes were randomly sample from the human genome to create sets of sizes
    given by \(x\) (in grey). The mean codon usage of the sets was calculated,
    and their Pearson correlation to the genomic background is shown on the
    \(y\) axis. Overlaid are actual gene sets given by human \go categories.}

The plot illustrates that few \go categories, if any, can confidently be said to
have a codon usage varying more than just randomly. In addition, the allocation
of individual \go categories to either set can certainly also be criticised: It
is certainly not clear why “translation” should be more active during cellular
division than in stable cells. The authors rather describe the relevant set as
“cell autonomous” \go terms, but the paper’s argument implicitly assumes that
this corresponds to cell division. Furthermore, the allocation of genes to \go
terms was performed via simple textual matching, such that \go sub-categories
whose description contains the text “differentiation and proliferation” would be
counted as belonging to the \go super-category “differentiation”.

In order to test whether the anticodon pool does indeed adapt to specific
cellular conditions, we tested whether the matching codon--anticodon adaptation
(estimated via the translation efficiency index \tai) is higher between matching
\mrna and \trna transcriptomes than between mismatching ones. In other words, we
looked at \mrna specific to cellular conditions, and calculated the \tai[s] for
\begin{enumerate*}
    \item matching \trna pools, and
    \item mismatching \trna pools
\end{enumerate*}
(\cref{fig:tai-matching-mismatching}).

\textfig{tai-matching-mismatching}{spill}{0.8\textwidth}
    {\tai between matching and mismatching codon--anticodon pools.}
    {}

\todo[inline]{What is the result?}

Furthermore, the trend seen in \cref{fig:go-cub-pca} does in fact exist to the
same degree when plotting amino acid usage rather than codon usage
(\cref{fig:go-aa-pca}). This suggests that rather than being driven by \go term
function, both codon usage and amino acid usage changes are driven by some other
genomic feature. In fact, the first principal component in \cref{fig:go-cub-pca}
correlates almost perfectly with \gc bias (\cref{fig:cub-pc1-vs-gc}). The nature
of this relationship is still unclear, and needs to be explored in more detail.

\textfig{go-aa-pca}{body}{0.8\textwidth}
    {\pca of mean \go term amino acid usage.}
    {\todo{Add description}}

\textfig{cub-pc1-vs-gc}{body}{0.8\textwidth}
    {\gc bias against PC1 of the mean \go term codon usage bias \pca.}
    {\todo{Add description}}

Furthermore, if a mechanism leading to the adaptation of the cellular \trna pool
to different cellular conditions existed, we would expect this to be well
conserved across mammalian evolution. Indeed, such an effect should show a
\emph{stronger} conservation of the codon usage bias across different functional
gene categories than conservation of other genomic features, such as the \gc
bias. Using genome data from different mammalian species and homology
information for the annotated \go categories in humans, we can calculate the
respective conservation of codon usage on the one hand, and \gc bias on the
other hand. The result is summarised in \cref{fig:cub-conservation}.

\textfig{cub-conservation}{body}{0.8\textwidth}
    {Conservation of \go term specific codon bias}
    {}
