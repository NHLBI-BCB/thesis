\clearpage
\phantomsection
\addcontentsline{toc}{part}{APPENDIX}
\part*{Appendix}
\label{sec:appendix}

\setcounter{chapter}{0}
\renewcommand\thechapter{\Alph{chapter}}

\chapter{Supplementary material for \texorpdfstring{\cref*{sec:trna}}{chapter 2}}

The material in this section has been taken from the supplementary figures \&
methods of Schmitt, Rudolph,\andothersdelim\& al. [\cite*{Schmitt:2014}] with
minimal changes to the figure legends. The figures and their captions have been
created jointly by Bianca Schmitt, Claudia Kutter and me.

The code used in the analysis of the data for this chapter can be found at
\url{https://github.com/klmr/trna} and
\url{https://github.com/klmr/trna-chip-pipeline}.

\begingroup
\renewcommand*\floatpos{H}

\textfig{trna-workflow}{spill}{\textwidth}
    {Workflow of the genome-wide identification and analysis of protein-coding
    and \trna genes.}
    {(A) \rnaseq analysis of protein-coding gene expression, differential
    expression analysis and codon usage analysis. (B) \chipseq analysis of \pol3
    occupancy at \trna gene loci, differential expression analysis of \trna
    genes, and anticodon isoacceptor abundance analysis.}

\textfig{mrna-heatmap}{spill}{\textwidth}
    {Hierarchical clustering of \mrna gene expression correlations.}
    {The heatmap shows the Spearman correlations of \mrna gene expression values,
    representing the same data as \cref{fig:mrna-pca}. The samples cluster
    hierarchically by tissue, followed by developmental stage.}

\textfig{trna-heatmap}{spill}{\textwidth}
    {Hierarchical clustering of \trna gene expression correlations.}
    {The heatmap shows the Spearman correlations of \trna gene expression values,
    representing the same data as \cref{fig:trna-pca}. The samples cluster
    hierarchically by tissue, followed by developmental stage, with few
    exceptions.}

\textfig{correlation-plots-rnaseq-pol3}{text}{0.7\textwidth}
    {Correlation of \rnaseq and \pol3 \chipseq data during mouse liver and brain
    development.}
    {Correlation of (A) protein-coding gene expression across developmental
    stages, (B) \trna gene expression as measured by \pol3 occupancy, (C)
    triplet codon usage in protein-coding genes, (D) \trna anticodon
    isoacceptor, (E) amino acid usage of protein-coding genes and (F) \trna
    amino acid isotype.}

\textfig{pca-all-stages}{spill}{\textwidth}
    {Early developmental stage-specific \trna genes are lowly expressed.}
    {(A) Factorial map of the \pca of \pol3 occupied \trna gene expression
    levels in liver (red), brain (yellow), embryonic body without head (light
    red) and head (light yellow) of stage E12.5, as well as whole E9.5 embryo
    (grey). The proportion of variance explained by the \abbrsc{PC} is indicated
    in parenthesis.\\
    (B) Violin plots represent normalized enrichment of \pol3 at \trna genes
    identified in E9.5 whole embryo (top), E12.5 head (middle) and E12.5 body
    without head (bottom) tissue. In parentheses are the numbers of \trna genes
    transcribed in the particular embryonic stage (“total \(>10\)”), which are
    subdivided into \trna genes that can be found in the \num{12} developmental
    stages according to \cref{fig:trna-counts} (“all tissues”) and those that
    are specific for the embryonic stage (“specific”).}

\textfig{codon-usage-mrna}{spill}{\textwidth}
    {Observed codon usage in \mrna transcriptomes of developing mouse liver.}
    {Proportional frequencies (\rcu) weighted by transcript expression are shown
    for triplet codons ordered by amino acid as a bar plot, where grey shading
    is by triplet codon. Data is obtained from liver \rnaseq data of all \num{6}
    developmental stages.}

\textfig{anticodon-abundance-trna}{spill}{\textwidth}
    {Observed anticodon abundance of \trna isoacceptors of developing mouse
    liver.}
    {Proportional frequencies weighted by \trna gene expression (\raa) are shown
    for anticodon isoacceptors ordered by amino acid isotype as a bar plot,
    where grey shading is by anticodon. Data is obtained from liver \pol3
    \chipseq data of all \num{6} developmental stages.}

\thispagestyle{empty}
\textfig{rnaseq-pol3-aa-usage-liver}{spill}{\textwidth}
    {Observed and simulated amino acid and isotype usage in transcriptomes
    across mouse liver development.}
    {Each panel (A–C) consists of three columns: experimentally observed data
    (left), simulated patterns of transcription randomized among either the
    expressed genes (middle) or all genomically encoded genes (right).
    Transcriptomes of each developmental stage were simulated \num{100} times.
    Proportional frequencies weighted by transcript expression are shown for (A)
    \num{20} amino acids as a radial plot, where data lines are coloured by
    developmental stage and the background of all genomically annotated \mrna
    genes is in grey. Labels within grid of radial plot describe ratios.
    Proportional frequencies weighted by \pol3 binding are shown for (B)
    \num{20} isotypes as a radial plot, both coloured as above (grey: background
    of all genomically annotated \trna genes). (C) Plot right panel shows
    Spearman’s rank correlation coefficients (\(\rho\)) and \(p\)-values (\(p\))
    of \pol3 binding to \trna isotypes (\(x\)-axis) and transcriptomic amino
    acid frequencies weighted by expression obtained from \rnaseq data
    (\(y\)-axis) in E15.5 liver (experimentally observed data) and all six
    developmental stages (simulated data). Amino acid isotypes outside the
    \num{99} per cent confidence interval (grey area within plot in C right) are
    named. Observed Spearman’s rank correlation coefficients across all stages
    (coloured as above) are indicated by black diamonds in plot C middle and
    left panels.}

\textfig{rnaseq-pol3-aa-usage-brain}{spill}{\textwidth}
    {\mrna codon usage and \pol3 occupancy of \trna isotypes in developing mouse
    brain tissue.}
    {Proportional frequency weighted by transcript expression of (A) arginine
    triplet codons, (B) amino acids, (C) \pol3 binding of arginine isoacceptors
    and (D) \pol3 binding of amino acid isotypes. Grey shading is by triplet
    codon (A) or \trna anticodon (C). Labels within grid of radial plot describe
    proportions.}

\textfig{codon-usage-low-vs-high-expressed-genes}{spill}{\textwidth}
    {Highly versus lowly expressed protein-coding genes show no differential
    codon usage.}
    {Proportional frequencies weighted by transcript expression are shown for
    arginine triplet codons as a bar plot of (A) highly (\nth{90}–\nth{95}
    percentile) and (B) lowly expressed (\nth{25}–\nth{50} percentile)
    protein-coding genes during liver development, where grey shading is by
    triplet codon. Plots show Spearman’s rank correlation coefficients
    (\(\rho\)) and \(p\)-values (\(p\)) of \pol3 binding to \trna isoacceptors
    (\(x\)-axis) and transcriptomic codon frequencies weighted by expression
    obtained from \rnaseq data (\(y\)-axis) in E15.5 liver of (C) highly and (D)
    lowly expressed protein-coding genes. Anticodon isoacceptors (grey dots in
    plots) are not encoded in the mouse genome and were excluded from
    calculating the correlation coefficients. (E) Variances of correlation
    values over all stages in liver (i) all expressed protein-coding genes, (ii)
    highly and (iii) lowly expressed protein-coding gene sets.}

\textfig{transcriptomic-pol3-codon-usage}{spill}{\textwidth}
    {Transcriptomic \mrna codon usage and \pol3 binding to \trna isoacceptors
    correlate in developing mouse liver and brain.}
    {Plots show correlation of proportional \pol3 binding to \trna isoacceptors
    (\(x\)-axis) and transcriptomic codon frequencies weighted by expression
    obtained from \rnaseq data (\(y\)-axis). Correlation plots for developing
    liver (A–F) and brain (G–L) are shown. Indexed box in top left indicates
    developmental stage. Grey dots represent degenerated codons. Spearman’s rank
    correlation coefficients (\(\rho\)) are reported along with their \(p\)-values
    (\(p\)) in bottom right of each panel.}

\thispagestyle{empty}
\textfig{codon-anticodon-correlation-with-wobble-only-missing}{spill}{\textwidth}
    {Transcriptomic \mrna codon usage and wobble corrected \pol3 binding to
    \trna isoacceptors correlate in developing mouse liver and brain.}
    {Plots show correlation of proportional \pol3 binding to \trna isoacceptors
    corrected according to wobble pairing (\(x\)-axis) and transcriptomic codon
    frequencies weighted by expression obtained from \rnaseq data (\(y\)-axis).
    Correlation plots for developing liver (A–F) and brain (G–L) are shown.
    Indexed box in top left indicates developmental stage. Spearman’s rank
    correlation coefficients (\(\rho\)) are reported along with their
    \(p\)-values (\(p\)) in bottom right of each panel.}

\thispagestyle{empty}
\textfig{transcriptomic-pol3-aa}{spill}{\textwidth}
    {Transcriptomic \mrna amino acid usage and \pol3 binding to \trna isotypes
    correlate in developing mouse liver and brain.}
    {Plots show correlation of \pol3 binding to \trna isotypes (\(x\)-axis) and
    transcriptomic amino acid frequencies weighted by expression obtained from
    \rnaseq data (\(y\)-axis). Correlation plots for developing liver (A–F) and
    brain (G–L) are shown. Indexed box in top left indicates developmental
    stage. Grey area represents \num{99} per cent confidence interval.
    Spearman’s rank correlation coefficients (\(\rho\)) and the corresponding
    \(p\)-values (\(p\)) are reported in top left and bottom right, respectively
    of each panel. Amino acid isotypes outside the \num{99} per cent confidence
    interval (grey area) are named.}

\textfig{colocalisation-e155-p22}{spill}{\textwidth}
    {Differentially expressed \trna genes show no colocalisation with
    differentially expressed protein-coding genes.}
    {In each plot, the blue line is the cumulative distribution of the ratio of
    the number of upregulated \mrna genes to the number of all \mrna genes in
    the neighbourhood of each upregulated \trna gene. The green line is the
    cumulative distribution of the ratios of the number of upregulated \mrna
    genes (\fdr cutoff \num{0.01}) to the number of all \mrna genes, in the
    neighbourhood of each \trna gene that is not differentially expressed.
    Significant differences between these two distributions reveal situations
    where upregulated \trna genes are significantly (by Kolmogorov–Smirnov test)
    associated with upregulated protein-coding genes. Different window sizes
    were used, ranging from \SIlist{10;50;100}{kb} around \trna genes. Pairwise
    comparison of (A–C) E15.5 and P22 in liver as well as (D–F) P4 and P29 in
    brain are shown. This analysis was repeated using two additional \fdr
    cutoffs (\num{0.05} and \num{0.}, data for liver in
    \cref{tab:colocalisation-liver}, not shown for brain). Under the assumption
    that there was an observable colocalisation effect, we would expect there to
    be a robust signal, i.e.\ consistent significance across different tested
    parameters. However, of the \num{18} tests, only one was significant
    (corrected \(p<0.013\)), after correcting for multiple testing (Bonferroni),
    indicating the absence of any strong localisation effect.}

\chapter{Supplementary material for \texorpdfstring{\cref*{sec:codons}}{chapter 3}}

\chapter{Supplementary material for \texorpdfstring{\cref*{sec:pol3}}{chapter 4}}

\endgroup
