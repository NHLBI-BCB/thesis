\clearpage
\phantomsection
\addcontentsline{toc}{part}{APPENDIX}
\part*{Appendix}
\label{sec:appendix}

\setcounter{chapter}{0}
\renewcommand\thechapter{\Alph{chapter}}

\chapter{Supplementary material for \texorpdfstring{\cref*{sec:trna}}{chapter 2}}

The material in this section has been taken from the supplementary figures \&
methods of Schmitt, Rudolph,\andothersdelim\& al. [\cite*{Schmitt:2014}] with
minimal changes to the figure legends. The figures and their captions have been
created jointly by Bianca Schmitt, Claudia Kutter and me.

The code used in the analysis of the data for this chapter can be found at
\url{https://github.com/klmr/trna} and
\url{https://github.com/klmr/trna-chip-pipeline}.

\begingroup
\renewcommand*\floatpos{H}

\textfig{trna-workflow}{spill}{\textwidth}
    {Workflow of the genome-wide identification and analysis of protein-coding
    and \trna genes.}
    {(A) \rnaseq analysis of protein-coding gene expression, differential
    expression analysis and codon usage analysis. (B) \chipseq analysis of \pol3
    occupancy at \trna gene loci, differential expression analysis of \trna
    genes, and anticodon isoacceptor abundance analysis.}

\textfig{correlation-plots-rnaseq-pol3}{text}{0.7\textwidth}
    {Correlation of \rnaseq and \pol3 \chipseq data during mouse liver and brain
    development.}
    {Correlation of (A) protein-coding gene expression across developmental
    stages, (B) \trna gene expression as measured by \pol3 occupancy, (C)
    triplet codon usage in protein-coding genes, (D) \trna anticodon
    isoacceptor, (E) amino acid usage of protein-coding genes and (F) \trna
    amino acid isotype.}

\textfig{pca-all-stages}{spill}{\textwidth}
    {Early developmental stage-specific \trna genes are lowly expressed.}
    {(A) Factorial map of the \pca of \pol3 occupied \trna gene expression
    levels in liver (red), brain (yellow), embryonic body without head (light
    red) and head (light yellow) of stage E12.5, as well as whole E9.5 embryo
    (grey). The proportion of variance explained by the \abbrsc{PC} is indicated
    in parenthesis.\\
    (B) Violin plots represent normalized enrichment of \pol3 at \trna genes
    identified in E9.5 whole embryo (top), E12.5 head (middle) and E12.5 body
    without head (bottom) tissue. In parentheses are the numbers of \trna genes
    transcribed in the particular embryonic stage (“total \(>10\)”), which are
    subdivided into \trna genes that can be found in the \num{12} developmental
    stages according to \cref{fig:trna-counts} (“all tissues”) and those that
    are specific for the embryonic stage (“specific”).}

\textfig{codon-usage-mrna}{spill}{\textwidth}
    {Observed codon usage in \mrna transcriptomes of developing mouse liver.}
    {Proportional frequencies (\rcu) weighted by transcript expression are shown
    for triplet codons ordered by amino acid as a bar plot, where grey shading
    is by triplet codon. Data is obtained from liver \rnaseq data of all \num{6}
    developmental stages.}

\textfig{anticodon-abundance-trna}{spill}{\textwidth}
    {Observed anticodon abundance of \trna isoacceptors of developing mouse
    liver.}
    {Proportional frequencies weighted by \trna gene expression (\raa) are shown
    for anticodon isoacceptors ordered by amino acid isotype as a bar plot,
    where grey shading is by anticodon. Data is obtained from liver \pol3
    \chipseq data of all \num{6} developmental stages.}

\chapter{Supplementary material for \texorpdfstring{\cref*{sec:codons}}{chapter 3}}

\chapter{Supplementary material for \texorpdfstring{\cref*{sec:pol3}}{chapter 4}}

\endgroup
